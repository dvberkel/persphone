\section*{Werkervaring}

	\begin{workExperience}{Topicus Zorg, Deventer}{Software ontwikkelaar}{\period{\moment{Juni}{2009}}{\present}}
		
		Als software ontwikkelaar heeft Daan voor Topicus Zorg de volgende
		activiteiten ontwikkeld
		
		\activity{Protopics HAP}
		
		\activity{Stage begeleiding}
		
		\activity{Promotie}
	\end{workExperience}

	\begin{workExperience}{Sogyo B.V., de Bilt}{Software ontwikkelaar}{\period{\moment{Juni}{2008}}{\moment{Juni}{2009}}}
		Als software ontwikkelaar heeft Daan voor Sogyo de volgende 
		activiteiten ontwikkeld.
		
		\activity{Pilot Masterclass Agile Software Ontwikkeling}
		Daan heeft meegedraaid in een pilot voor Agile software ontwikkeling. In
		de pilot werd een ontwikkelstraat ingericht voor een bibliotheek 
		systeem. Er is aandacht besteed aan	een source repository, continous 
		integration, project communicatie middels trac, de ontwikkelmethodiek 
		SCRUM en het documenteren, overdragen en opleveren van software.
		Daan heeft door zijn eerdere ervaringen met SCRUM een leidende rol
		genomen in het bekend maken van zijn team met deze vorm van agile 
		software ontwikkelen.
		\technics o.a. Java, TestNG, Seam, JSF, Ant, svn, TeamCity, Trac, Agilo.
		
		\activity{Sogyo Interactieve Grafieken}
		Dit project omvat de bouw van een projectplanner. De scope van dit 
		project is het toekennen van personen aan projecten met een bepaalde 
		taak. Daan heeft zich in dit project een nieuwe technologie voor RIA's 
		eigen gemaakt, namelijk JavaFX. Met deze technologie is een client 
		geschreven die interfaced met een in dit project ontwikkeld java backend
		systeem. In dit project is SCRUM toegepast als ontwikkelmethodiek. Het 
		gebruik van SCRUM tijdens dit project is een succes gebleken en wordt nu
		voor de meeste projecten binnen Sogyo gebruikt.
		\technics o.a. Java, JavaFX, JUnit, XML, XSLT, XPath, Tomcat, HTML, 
		JavaScript, CSS, JSP, Servlets, Maven, Perl, Latex, Trac, Agilo, Flex, 
		C\#, svn, Hudson, SCRUM.

		\activity{Internetbankieren}
		In het ``internetbankieren-project'' wordt een webapplicatie gebouwd voor
		het online-afhandelen van typische bankzaken zoals aanmaken van nieuwe 
		accounts, boekingen en het opvragen van rekeningoverzichten. Hierbij is
		de reeds bestaande back-end die in het bankierenproject is opgeleverd, 
		ge\"integreerd met een nieuw ontworpen front-end. De front-end bestaat uit
		zowel pagina\'s als formulieren. Er is met name aandacht besteed aan
		security en de loginprocedure voor drie typen gebruikers: clienten, 
		beheerders en passanten. Binnen de security wordt ook ruim aandacht 
		besteed aan de rollen die de gebruikers spelen en welke rechten zij
		binnen een rol hebben.
		\technics J2EE, Servlets, JSP, Tomcat web server, CSS en HTML. 
		
		\activity{Mancala}
		Voor het ``Mancala'' project wordt er een elektronische versie van het 
		Afrikaanse bordspel Mancala in Java gebouwd. De scope van het 
		project is afgebakend op de implementatie van de spelregels. Tijdens
		de ontwikkeling ligt de nadruk op het maken van een 
		object-geori\"enteerd domeinmodel. Daan was als ontwerper/ontwikkelaar
		verantwoordelijk voor de ontwikkeling en de implementatie van het 
		spel.
		\technics UML, Java SE en Eclipse.
		
		\activity{Bankieren}
		Daan heeft meegewerkt aan de bouw van een basissysteem voor het 
		afhandelen van administratieve handelingen binnen een bank. Klanten 
		moeten verschillende soorten rekeningen kunnen aanmaken (zoals betaal- 
		en spaarrekeningen), stortingen en opnamen kunnen verrichten en 
		transacties uitvoeren. Daan was tevens verantwoordelijk voor het 
		implementeren van deze applicatie. 
		Er is vooral aandacht besteed aan de standaard-onderdelen van de Java 
		Software Development Kit zoals objectori\"entatie, threading, collection 
		framework, flow control en exception handling.
		\technics Java SE, Eclipse. 
	\end{workExperience}
	
	\begin{workExperience}{SumOfUs, Nijmegen}{Software Ontwikkelaar}{\period{\moment[01]{07}{2006}}{\moment[01]{08}{2006}}}
		Daan heeft met behulp van PostScript een script geschreven dat direct 
		naar de printer gestuurd kon worden, dit ter vervanging van honderden 
		stapels scoreblaadjes die handmatig geschreven moesten worden.
		\technics o.a. Perl, Postscript.
	\end{workExperience}

	\begin{workExperience}{Ratio, Nijmegen}{Software Ontwikkelaar}{\period{\moment[01]{01}{2001}}{\moment[01]{01}{2005}}}
		In dit project heeft Daan de automatisering van het op het internet 
		beschikbaar maken van een uitdagende wiskundelesmethode gemaakt. Hierbij
		was het wenselijk dat leerlingen online opgaven konden maken die 
		automatisch gecontroleerd werden.
		\technics o.a. Perl, HTML, CSS, JavaScript.	
	\end{workExperience}
