\section*{Personalia}

	\credentials

\section*{Algemeen}

Daan is ontzettend leergierig, enthousiast, gestructureerd, analytisch en breed 
ge\"interesseerd. Daan is in teamverband zeer sociaal en prettig in de omgang. 
Hij is behulpzaam en draagt bij aan een goede sfeer binnen het team. Zijn 
interesse in het software ontwikkelen heeft hij voortgezet in zijn studie 
wiskunde door zich te specialiseren in de computeralgebra en bijdragen te 
leveren aan het computeralgebra-pakket MAGMA. Daarnaast heeft Daan zich buiten 
zijn studie op eigen initiatief allerlei ontwikkelkennis eigen gemaakt, zoals 
Perl, PostScript en Haskell.

\section*{Opleidingen}

	\begin{educationList}
		\item[\period{\moment{07}{2008}}{\moment{11}{2008}}]%
		Mastercourse Software Engineering, Sogyo, de Bilt.
		\item[\period{\moment{07}{2000}}{\moment{07}{2008}}]%
		Wiskunde, Radboud Universiteit, Nijmegen.
		\item[\period{\moment{07}{1992}}{\moment{07}{1999}}]%
		VWO, Candea College, Zevenaar.
	\end{educationList}

\section*{Cursussen}

	\begin{courseList}
		\item[\period{\moment{10}{2008}}{\moment{11}{2008}}]%
		SCWCD 5: Sun Microsystems, Sun Certified Web Component Developer for the
		Java Platform, Enterprise Edition 5.
		\item[\period{\moment{09}{2008}}{\moment{10}{2008}}]%
		 OOF: EXIN Foundation Certificate in Object Orientation, behaald.
		\item[\period{\moment{07}{2008}}{\moment{09}{2008}}]%
		SCJP 5: Sun Microsystems, Sun Certified Programmer for the Java 
		Platform, Standard Edition 5, behaald.
	\end{courseList}

\section*{Werkervaring}

	\begin{workExperience}{Sogyo B.V., de Bilt}{Software ontwikkelaar}{\period{\moment{Juni}{2008}}{\present}}
		Als software ontwikkelaar heeft Daan voor Sogyo de volgende 
		activiteiten ontwikkeld.
		
		\activity{Pilot Masterclass Agile Software Ontwikkeling}
		Daan heeft meegedraaid in een pilot voor Agile software ontwikkeling. In
		de pilot werd een ontwikkelstraat ingericht voor een bibliotheek 
		systeem. Er is aandacht besteed aan	een source repository, continous 
		integration, project communicatie middels trac, de ontwikkelmethodiek 
		SCRUM en het documenteren, overdragen en opleveren van software.
		Daan heeft door zijn eerdere ervaringen met SCRUM een leidende rol
		genomen in het bekend maken van zijn team met deze vorm van agile 
		software ontwikkelen.
		\technics o.a. Java, TestNG, Seam, JSF, Ant, svn, TeamCity, Trac, Agilo.
		
		\activity{Sogyo Interactieve Grafieken}
		Dit project omvat de bouw van een projectplanner. De scope van dit 
		project is het toekennen van personen aan projecten met een bepaalde 
		taak. Daan heeft zich in dit project een nieuwe technologie voor RIA's 
		eigen gemaakt, namelijk JavaFX. Met deze technologie is een client 
		geschreven die interfaced met een in dit project ontwikkeld java backend
		systeem. In dit project is SCRUM toegepast als ontwikkelmethodiek. Het 
		gebruik van SCRUM tijdens dit project is een succes gebleken en wordt nu
		voor de meeste projecten binnen Sogyo gebruikt.
		\technics o.a. Java, JavaFX, JUnit, XML, XSLT, XPath, Tomcat, HTML, 
		JavaScript, CSS, JSP, Servlets, Maven, Perl, Latex, Trac, Agilo, Flex, 
		C\#, svn, Hudson, SCRUM.

		\activity{Internetbankieren}
		In het ``internetbankieren-project'' wordt een webapplicatie gebouwd voor
		het online-afhandelen van typische bankzaken zoals aanmaken van nieuwe 
		accounts, boekingen en het opvragen van rekeningoverzichten. Hierbij is
		de reeds bestaande back-end die in het bankproject is opgeleverd, 
		geïntegreerd met een nieuw ontworpen front-end. De front-end bestaat uit
		zowel pagina’s als formulieren. Er is met name aandacht besteed aan
		security en de loginprocedure voor drie typen gebruikers: clienten, 
		beheerders en passanten. Binnen de security wordt ook ruim aandacht 
		besteed aan de rollen die de gebruikers spelen en welke rechten zij
		binnen een rol hebben.
		\technics J2EE, Servlets, JSP, Tomcat web server, CSS en HTML. 
		
		\activity{Mancala}
		Voor het ``Mancala'' project wordt er een elektronische versie van het 
		Afrikaanse bordspel "Mancala" in Java gebouwd. De scope van het 
		project is afgebakend op de implementatie van de spelregels. Tijdens
		de ontwikkeling ligt de nadruk op het maken van een 
		object-geori\"enteerd domeinmodel. Daan was als ontwerper/ontwikkelaar
		verantwoordelijk voor de ontwikkeling en de implementatie van het 
		spel.
		\technics UML, Java SE en Eclipse.
		
		\activity{Bankieren}
		Daan heeft meegewerkt aan de bouw van een basissysteem voor het 
		afhandelen van administratieve handelingen binnen een bank. Klanten 
		moeten verschillende soorten rekeningen kunnen aanmaken (zoals betaal- 
		en spaarrekeningen), stortingen en opnamen kunnen verrichten en 
		transacties uitvoeren. Daan was tevens verantwoordelijk voor het 
		implementeren van deze applicatie. 
		Er is vooral aandacht besteed aan de standaard-onderdelen van de Java 
		Software Development Kit zoals objectori\"entatie, threading, collection 
		framework, flow control en exception handling.
		\technics Java SE, Eclipse. 
	\end{workExperience}
	
	\begin{workExperience}{SumOfUs, Nijmegen}{Software Ontwikkelaar}{\period{\moment[01]{07}{2006}}{\moment[01]{08}{2006}}}
		Daan heeft met behulp van PostScript een script geschreven dat direct 
		naar de printer gestuurd kon worden, dit ter vervanging van honderden 
		stapels scoreblaadjes die handmatig geschreven moesten worden.
		\technics o.a. Perl, Postscript.
	\end{workExperience}

	\begin{workExperience}{Ratio, Nijmegen}{Software Ontwikkelaar}{\period{\moment[01]{01}{2001}}{\moment[01]{01}{2005}}}
		In dit project heeft Daan de automatisering van het op het internet 
		beschikbaar maken van een uitdagende wiskundelesmethode gemaakt. Hierbij
		was het wenselijk dat leerlingen online opgaven konden maken die 
		automatisch gecontroleerd werden.
		\technics o.a. Perl, HTML, CSS, JavaScript.	
	\end{workExperience}

\section*{Nevenactiviteiten}

	\begin{subActivityList}
		\item[Weblog] Op \url{http://www.software-innovators.nl/}, de corporate
		weblog van Sogyo, heeft Daan een actief blog. Op deze blog publiceert 
		hij zijn gedachten op het gebied van software ontwikkeling.
		\hfill\\
		
		\item[Begeleiding Winnaars] De Radboud Universiteit organiseert 
		jaarlijks een wiskunde toernooi voor leerlingen van de middelbare 
		school. In 2007 was de hoofdprijs een reis naar New York. De twee 
		winnende teams hebben onder begeleiding een trip langs het financi\"ele 
		district in New York gemaakt. Daan was \'e\'en van de vier begeleiders.
		\hfill\\
		
		\item[Essay] ``Trots op mijn studie'' is een bundeling van twaalf essays 
		(ISBN: 9789058750884). Daan laat in een van die twaalf essays zien op 
		welke wijze en in welk mate hij betrokken is bij de waarden van de 
		wetenschap.
		Een licht ingekorte versie van Daan zijn essay is verschenen in het 
		``Nieuw Archief voor Wiskunde''.		
		\hfill\\
	\end{subActivityList}

\section*{Technische Vaardigheden}

	\begin{skillList}
		\item[Talen] Adobe Flex, BASIC, C, C\#, C++, Groovy, Java, JavaFX,
		JavaScript, MetaPost, Pascal, Perl, PHP, PostScript, Python, Ruby.
		\hfill\\
		
		\item[Technologi\"en] Ant, Checkstyle, CSS, DTD / XSD, Eclipse, Glassfish, 
		HTML / XHTML, Hudson, JavaServer Faces (JSF), JavaServer Pages (JSP), 
		Java Standard Tag Library (JSTL), JBoss, JCoverage, JUnit, Latex, Maven,
		MySQL, Netbeans, Wordpress, PMD, Seam, Sendmail, Servlets, Subversion,
		TestNG, Tomcat, Trac, Trac / Agilo, Unified Modelling Language (UML), XML,
		XPath, XSLT.
		\hfill\\
		
		\item[Methoden] Agile Software Ontwikkeling, Design Patterns Basic 
		(GoF), Domain Driven Design, eXtreme Programming (XP), Intergration testing,
		Object Orientatie (OO), Scrum, Test Driven Design, Unit testing.
		\hfill\\
		
	\end{skillList}

\section*{Talen}

	\begin{languageList}
		\item[Nederlandse taal] Goede beheersing in woord en geschrift.
		\item[Engelse taal] Goede beheersing in woord en geschrift.
	\end{languageList}

\section*{Hobby's}
	
	Aikido, (rots)klimmen, salsadansen


